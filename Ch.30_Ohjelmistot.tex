\chapter{Ohjelmistokirjastoja rakenteen oppimiseen\label{software}}
\citet{scanagatta_survey_2019} esittävät katsauksessaan kattavan yhteenvedon erilaisista ohjelmistoista Bayes-verkkojen käsittelmiseen ja rakenteen oppimiseen. Heidän esittelynsä kattaa niin kaupallisia, kuin vapaasti saatavilla olevia ohjelmistoja ja ohjemistokirjastoja. Tässä luvussa esittelen ja vertailen kolmea hyvin dokumentoitua, vapaasti saatavilla olevaa ja aktiivisesti kehityksessä olevaa ohjelmistokirjastoa suosituimmillelle data-analyyseissä käytetyille ohjelmointikielille. Yhteenveto kirjastoista on esitetty taulukossa ~\ref{table:softat}.

\begin{table}[ht]
\centering
\caption{Ohjelmistokirjastoja Bayes-verkkojen käsittelemiseen\label{table:softat}}
\begin{tabular}{l|| c c c c} 
kirjasto & kieli & versio & päivitetty & lisenssi \\
\hline \hline 
\href{https://www.bnlearn.com/}{bnlearn} & R & 4.8.1 & 2022-09-21 & CC BY-SA 3.0 \\
\href{https://github.com/davenza/PyBNesian}{PyBNesian} & Python & 0.4.2 & 2022-03-26 & MIT \\
\href{https://github.com/sisl/BayesNets.jl}{BayesNets.jl} & Julia & 3.4.0 & 2022-07-06 & MIT  \\
\hline
\end{tabular}
\end{table}
\section{bnlearn}

\section{PyBNesian}

\section{BayesNets.jl}


