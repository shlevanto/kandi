\chapter{Pohdinta: Bayes-verkkojen sovellusmahdollisuudet \label{conclusions}}
Yleisesti Bayes-verkon rakenteen oppimisella voidaan ajatella olevan neljä eri sovelluskohdetta: 1) graafin tulkitseminen muuttujien välisten yhteyksien ymmärtämiseksi 2) posterioritapahtuman ennustaminen, 3) prioritapahtumien ennustaminen, kun posterioritapahtuma on tunnettu ja 4) intervention kohdentaminen sopivaan kohtaan. Tämä intervention kohdentaminen on merkittävästi muita sovelluskohteita haastavampi, koska se edellyttää tietoa muuttujien välisten riippuvuussuhteiden kausaalisuudesta. Kuten luvussa \ref{bayes-verkko} mainitaan, Bayes-verkko ei lähtökohtaisesti sisällä tätä tietoa. Kausaalisuhteet voidaan määritellä asiantuntijatiedon perusteella, kuten \citet{xu_workplace_2022} graafista mallia hyödyntävässä tutkimuksessa ja \citet{liu_empirical_2012} Bayes-verkkojen pisteytystä käsitelleessä tutkimuksessa tehtiin.

Bayes-verkkoilla on sovelluksia insinööritieteissä vikadiagnostiikassa ja luotettavuuden arvioinnissa \citep{zhang_brief_2019}, lääketieteessä diagnostiikan apuna \citep{mittal_review_2011} ja tietoverkoissa poikkeavien tapahtumien tunnistamisessa \citep{kaur_review_2013}.

Vakuutusalan osalta \citet{ramsahai_connecting_2020} esittelee graafisten mallien sovelluskohteina mm. tapahtumien frekvenssin ja vakavuuden välisen riippuvuuden tarkastelun, riippumattomien tekijöiden taustalla vaikuttavien välillisten muuttujien tunnistamisen lineaarisissa malleissa ja latenttien muuttujien tunnistamisen puutteellisessa datassa. Lisäksi muuttujien riippuvuuksien kartoittamisesta voisi hänen mukaansa olla hyötyä, kun halutaan selvittää minkälaisia riippuvuuksia erilaisten summamuuttujien, kuten luottoluokituksen, taustamuuttujien välillä on. 

Yksilötason työkyvyttömyysriskin ennustamiseen on jo käytettävissä laaja joukko todennettuja riskitekijöitä.  Näiden muuttujien yhteisvaikutuksia ei kuitenkaan ole tarkasteltu riittävästi. Bayes-verkkojen avulla voitaisiin parhaimmillaan paitsi tarkastella muuttujien välisiä riippuvuuksia, myös tunnistaa niitä muuttujia, joihin olisi mahdollista suunnata työkyvyttömyysriskiä alentavia interventioita. Tämä tosin vaatii myös muuttujien välisten kausaalisuhteiden tarkastelemista. 

Muuttujien välisten yhteyksien kuvaamista voisi myös käyttää työkyvyttömyysriskiä alentamaan tarkoitetun ammatillisen kuntoutuksen kehittämiseen. Mallintamalla onnistuneeseen tai epäonnistuneeseen kuntoutukseen liittyvien taustamuuttujien välisiä riippuvuussuhteita voitaisiin esimerkiksi tunnistaa sellaisia henkilöryhmiä, jotka hyötyisivät muita paremmin kuntoutuksesta. Vastaavasti voitaisiin tunnista sellaisia kuntoutusmuotoja, joiden ennuste tietylle hakijalle tai hakijaryhmälle olisi erityisen suotuisa. 

Suunnattuja graafimalleja on jo hyödynnetty muuttujien välisten yhteyksien kuvaamiseen työkyvyttömyysriskille läheisissä aiheissa, kuten \citet{elovainio_is_2021, elovainio_network_2020} tutkimuksissa masennuksen taustatekijöistä ja niiden keskinäisistä yhteyksistä sekä \citet{xu_workplace_2022} tutkimuksessa työpaikan voimavaratekijöiden suojaavasta vaikutuksesta sydän- ja verisuonitauteihin sairastumisen riskistä, jossa verkon uskottavuuden laskemiseen käytettiin BIC-arvoa vaikka verkko oli rakennettu asiantuntijatiedon pohjalta eikä sen rakennetta laskettu datasta.

Koneoppimismallien yleisesti tunnistettuna ongelmana on niiden läpinäkymättömyys. Mallit antavat datan pohjalta ennusteen tai luokittelun, mutta voi olla vaikea saada tietoa siitä miten malliin syötetyt muuttujat ovat vaikuttaneet lopputulokseen. \citet{adekunle_applied_2021} pohtivat eläkealan data-analytiikkaa käsittelevässä raportissaan tämän olevan yksi keskeinen syy siihen, miksi koneoppimismallien soveltamisessa ollaan alalla vasta alussa. Riippuvuuksien tunnistaminen niiden muuttujien osalta, joilla on koneoppimismalleissa suurin selitysarvo, on kuitenkin mielenkiintoinen tutkimussuunta. Koneoppimismalleissa käytettävien muuttujien painoarvoja mallin tuottamassa ennusteessa voidaan nimittäin tarkastella jälkikäteen peliteoriasta lähtöisin olevien Shapley-arvojen avulla \citep{shapley_7_2020}. Shapley-arvojen tarkempi käsittely rajautuu tämän tutkielman ulkopuolelle, mutta \citet{merrick_explanation_2020} tarjoavat kattavan johdannon koneoppimismallien arvioimiseen niiden avulla. 

Shapley-arvot eivät suoraan anna tulkintamahdollisuutta muuttujien välisistä riippuvuuksista, mutta niiden avulla voitaisiin valita Bayes-verkkoon mallissa voimakkaimmin vaikuttavat muuttujat. Muuttujat voitaisiin myös jakaa painoarvoltaan voimakkaampiin ja vähemmän voimakkaammin, ja tarkastella löytyykö pienemmän painoarvon muuttujista mielenkiintoisia latentteja muuttujia. Myös mallin muuttujien Shapley-arvojen käyttäminen satunnaismuuttujina Bayes-verkon rakentamisessa voisi tarjota mielenkiintoisia tulkintamahdollisuuksia koneoppimismalleista.

Bayes-verkkojen sovellusmahdollisuuksia työkyvyttömyysriskin mallintamisessa ei ole vielä tutkittu, mutta tämän tutkielman perusteella voidaan esittää, että Bayes-verkoilla ja erityisesti rakenteen oppimisen menetelmillä on useita tarkemman tarkastelun arvoisia sovelluskohteita työkyvyttömyysriskin mallintamisessa. 