\chapter{Pohdinta\label{conclusions}}
Lopullisen tutkielmani pohdinnassa käsittelen lyhyesti Bayes-verkon rakenteen oppimista ja mitä sillä voidaan saada aikaiseksi. Lisäksi käsittelen  tarkemmin lukua \ref{applications} tarkemmin sovellusmahdollisuuksia eläkedatassa. Meillä on hyvää pohjatietoa siitä, minkälaisia muuttujia voidaan käyttää eläkeriskin arvioimiseksi, mutta mitä saavutettaisiin sillä, että muuttujien väliset riippuvuudet tehdään näkyviksi? Voitaisiinko tunnistaa niitä Bayes-verkon polkuja, joihin ehkä olisi mahdollista tehdä interventioita? Tai voidaanko ainakin löytää sellaisia riippuvuusketjuja ,jotka tulisi huomioida vaikka kausaalisuhteita ei välttämättä tiedetä?

