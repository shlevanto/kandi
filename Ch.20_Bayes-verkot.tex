\chapter{Bayes-verkot\label{bayes}}

\section{Graafiset mallit}
Graafinen malli (\emph{graphical model}) yhdistelee tilastotiedettä, todennäköisyysteoriaa ja graafiteoriaa, ja kuvaa näiden avulla satunnaismuuttujia ja näiden välisiä yhteyksiä graafeina \citep{jordan_graphical_2004}. Sinällään graafisen mallin käsite ei ole kovin tarkkarajainen, vaan sillä voidaan tarkoittaa aineiston kuvaamista graafimuodossa myös ilman, että graafin muodostamiseen on käytetty laskentaa. \citet{ramsahai_connecting_2020} jaottelee graafiset mallit sellaisiin, jotka on muodostetettu kvalitatiivisesti ilman laskentaa ja sellaisiin, jotka on muodostettu kvantitatiivisesti laskennan avulla. 

\citet{jordan_graphical_2004} puolestaan määrittelee probabilistisen graafisen mallin sellaiseksi, jossa graafin solmut kuvaavat satunnaismuuttujia ja kaaret näiden välisiä riippuvuuksia. Tällaista graafista mallia kutsutaan \emph{Bayes-verkoksi} (\texttt{Bayesian network}), kun mallissa käytetään suunnattua syklitöntä graafia ja vastaavasti Markov-verkoksi, jos kaaret ovat suuntaamattomia \citep{ruggeri_bayesian_2008}. Ensimmäisenä Bayes-verkkojen käyttömahdollisuuksia on esittänyt \citet{pearl1986fusion}.

\section{Bayes-verkko}
Bayes-verkko on \emph{suunnattu syklitön verkko} (\texttt{Directed Acyclic Graph, DAG}), joka kuvaa satunnaismuuttujien yhteistodennäköisyysjakaumaa \citep{ruggeri_bayesian_2008}. Määritelmällisesti Bayes-verkko on pari 
$$
    B = \langle G, \Theta \rangle
$$
jossa 
$$
    G = \{V, E\}, V = \{X_1, X_2,\ldots X_n\}
$$ 
eli graafin solmut kuvaavat satunnaismuuttujia ja kaarien joukko $E$ kuvaa näiden välisiä suoria riippuvuuksia. Määritelmälllisesti Bayes-verkossa kukin muuttuja on riippumaton kaikista niistä muuttujista, jotka eivät ole tämän muuttujan jälkeläisiä kun tälle muuttujalle on määritelty verkossa sen edeltäjien muodostama joukko. Parametrijoukko $\Theta$ koostuu satunnaismuuttujien $X_i$ parametreista 
$$
    \theta_{x_i|\pi_i} = P_B(x_i|\pi_i)
$$ 
eli ehdollisista todennäköisyyksistä tapaukselle $x_i$ ehtonaan muuttujan $X_i$ edeltäjien joukko verkossa $G$ \citep{ruggeri_bayesian_2008, myllymaki_bayes-verkkojen_1998}. Joukon $\Theta$ alkiot voidaan esittää \emph{ehdollisten todennäköisyyksien taulukoina}  (\texttt{conditional probability table, CPT}) kaikille niille tapauksille, joita muuttujan edeltäjiensä ehdoilla voi saada \citep{zhang_brief_2019}.

Tarkastellaan kuvassa \ref{fig:bayes-esim} esitettyä yksinkertaista Bayes-verkkoa. Kesällä on varsin suuri todennäköisyys, että henkilö on hiljattain käynyt uimassa $P(U) = 0.8$. Toiveikkaasti ajatellen todennäköisyys että kesäpäivänä sataa on melko pieni $P(S) = 0.1$. 

\begin{center}
    \captionof{figure}{Kesäinen Bayes-verkko \label{fig:bayes-esim}}
    \tikzfig{bayes_ex_1}
\end{center}
    
Jos henkilö käy uimassa, on hänen tukkansa märkä, mutta paita kuiva. Jos sataa, sekä tukka että paita kastuvat eli. Tukan märkyys riippuu sekä sateesta että uinnista. Vastaavasti paidan märkyys riippuu vain sateesta. Taulukossa \ref{table:bayes-esim} on esitetty tämän esimerkkiverkon parametrijoukon alkiot.

\begin{center}

\captionof{table}{Esimerkkiverkon parametrijoukko $\Theta$ \label{table:bayes-esim}} 
\begin{tabular}{ cccc }   % top level tables, with 4 columns
    \begin{tabular}{||c ||} 
     \hline
     S \\
     \hline\hline
     $s^1 = 0.1$ \\ 
     $s^0 = 0.9$ \\
     \hline
    \end{tabular} &  
    
    % table 2
    \begin{tabular}{||c ||} 
     \hline
     U \\
     \hline\hline
     $u^1 = 0.8$ \\ 
     $u^0 = 0.2$ \\
     \hline
    \end{tabular} &
    
    % table 3
    \begin{tabular}{||c ||} 
     \hline
     P \\
     \hline\hline
     $p^1 | s^1 = 1$ \\ 
     $p^1 | s^0 = 0$ \\
     $p^0 | s^0 = 1$ \\
     $p^0 | s^1 = 0$ \\
     \hline
    \end{tabular} &
    
    % table 3
    \begin{tabular}{||c ||} 
     \hline
     T \\
     \hline\hline
     $t^1 | (s^1 \cap u^1) = 1$ \\ 
     $t^1 | (s^1 \cap u^0) = 1$ \\
     $t^0 | (s^0 \cap u^1) = 1$ \\
     $t^0 | (s^0 \cap u^0) = 0$\\
     \hline
    \end{tabular}
\end{tabular}
\end{center}
Nyt voimme esimerkiksi laskea todennäköisyyden sille, että satunnaisena kesäpäivänä henkilöllä on märkä tukka
$$
    P(T) = (0.1\cdot0.8) + (0.1\cdot0.2) + (0.9\cdot0.8) = 0.82
$$

Bayes-verkon avulla voidaan nyt paitsi hahmottaa, miten eri muuttujat ovat riippuvaisia toisistaan, myös laskea Bayesin kaavaa käyttäen prioritodennäköisyyksiä esim. todennäköisyys sille, että sataa kun tiedetään että henkilön tukka on märkä.

$$
  P(S|T) = \frac{P(T|S)P(S)}{P(T)} = \frac{1 \cdot P(0.1)}{P(0.82)} = 0.12   
$$

Bayes-verkko kuvaa siis satunnaismuuttujajoukon yhteistodennäköisyysjakaumaa ja siihen kuuluvien muuttujien välisiä riippuvuussuhteita. Vaikka Bayes-verkossa näitä riippuvuuksia kuvataan suunnattuina kaarina, painottavat mm. \citet{ruggeri_bayesian_2008, myllymaki_bayes-verkkojen_1998} että Bayes-verkon kaaret suuntineen eivät välttämättä kuvaa muuttujien välisiä kausaalisuhteita. Kausaalisuhteiden päätteleminen tai oppiminen datasta on toki mahdollista ja muodostaa oman mielenkiintoisen aihekokonaisuutensa. \citet{vowels_dya_2022} tarkastelee tuoreessa katsauksessaan tähän soveltuvia menetelmiä, mutta niiden käsitteleminen rajautuu tämän opinnäytetyön ulkopuolelle.

\section{Bayes-verkkojen pisteyttäminen}
Bayes-verkon \emph{kokonaisuskottavuutta} eli sopivuutta käsitellään mm. teoksessa \citep{myllymaki_bayes-verkkojen_1998} siten, että todennäköisin malli annetulla datajoukolla on paras. Kokonaisuskottavuutta arvioidaan ehdollisena posterioritodennäköisyytensä $P(M | \mathcal{D}$, missä $\mathcal{D}$ on mallin opettamiseen käytetty aineisto. Tunnetuin Bayes-verkon uskottavuutta kuvaava pisteytys on \citet{schwarz_estimating_1978} esittelemä \emph{BIC} (\texttt{Bayesian Information Criterion}) ja sitä vastaava \emph{MDL} (\texttt{Minimum Description Length}) \citep{ruggeri_bayesian_2008, liu_empirical_2012}. \citet{myllymaki_bayes-verkkojen_1998} esittävät BIC:n approksimaatioon kaavan

$$
P(\mathcal{D}|M) = \frac{P(\mathcal{D}|M, \bar{\theta})}{N{\frac{d(M)}{2}}}
$$

jossa $\bar{\theta}$ kuvaa suurimman uskottavuuden parametrijoukkoa, $d(M)$ malliparametrien lukumäärä ja $N$ mallin opettamiseen käytet datajoukon koko. MDL puolestaan on tämän approksimaation negatiivinen logaritmi (kts. \citet{rissanen_stochastic_1987}).

\section{Bayes-verkon rakenteen oppiminen}
Bayes-verkkojen käytännön sovelluksissa verkon rakenne on usein tuntematon \citep{ruggeri_bayesian_2008}, eli muuttujien välisiä yhteyksiä ei tiedetä. Tällöin mielenkiintoiseksi tehtäväksi muodostuu verkon rakenteen päätteleminen tai laskeminen datan pohjalta. \citet{myllymaki_bayes-verkkojen_1998} mukaan mallintamisessa edetään ensin verkon rakenteen oppimisen kautta parametrijoukon laskemiseen. Tämä Bayes-verkon rakenteen oppiminen on kompleksinen, todetusti NP-kova ongelma \citep{chickering_large-sample_2004}. On helppo nähdä, että jo yksinkertaisten dikotomisten muuttujien muodostamassa Bayes-verkossa on $2^v$ mahdollista rakennetta, kun $v$ on satunnaismuuttujien määrä ja muuttujat ovat kaikki jollakin tavoin keskenään riippuvaisia. Nähdään siis, että kaikkien mahdollisten verkkojen laskeminen johtaa pahimmassa tapauksessa eksponentiaalisesti kasvavaan joukkoon. Rakenteen oppimisen kompleksisuus tekeekin siitä erityisen mielenkiintoisen, koska sen ratkaiseminen vaatii sopivaa tasapainottelua tarkkuuden ja laskentatehovaatimusten välillä.

Rakenteen oppimista voidaan lähestyä kahdella tavalla: \emph{pisteyttämällä} (\texttt{score based}),  \emph{rajoitepohjaisesti} (\texttt{constraint based}) \citep{ramsahai_connecting_2020, scutari_learning_2010} tai yhdistämällä näitä esim. \citep{li_hybrid_2018}. Pisteyttämisessä etsitään eri vaihtoehtojen joukosta sellainen Bayes-verkko, jonka sopivuutta kuvaava funktio (esim. BIC) saa parhaan mahdollisen arvon. BIC:n ohella toinen suosittu pisteytys on \emph{BDeu} (\texttt{Bayesian Dirichlet equivalent uniform})  \citep{scanagatta_survey_2019}. Näiden lisäksi on olemassa erilaisia paikallisen rakenteen optimointiin perustuvia pisteytyksiä. Empiirisellä datalla  tehdyssä vertailussa \citet{liu_empirical_2012} ovat osoittaneet, että nimenomaan BIC ja sitä vastaava MDL antavat hyvin luotettavia tuloksia verkon rakenteen oppimisessa. 

Rajoitepohjaisessa lähestymistavassa puolestaan solmuissa oleville muuttujille ajetaan tilastollisia testejä (esim. $\chi^2$), joiden avulla arvioidaan muuttujien riippumattomuutta ja näin pyritään löytämään aineistoon sopivin tai uskottavin verkko \citep{ramsahai_connecting_2020, scutari_learning_2010}. 

Bayes-verkon rakennetta voidaan myös täydentää kvalitatiivisesti käyttämällä asiantuntijatietoa muuttujien välisistä suhteista \citep{ruggeri_bayesian_2008, myllymaki_bayes-verkkojen_1998}. 

\citet{mittal_review_2011} esittelevät yhteenvedon ja vertailun muutamasta yleisimmin käytetystä algoritmista Bayes-verkon rakenteen oppimiseksi. Lisäksi he myös vertailevat algoritmien tarkkuutta ja tehokkuutta erikokoisissa aineistoissa.

\citet{scanagatta_survey_2019} tarkastelevat tarkemmin erilaisia menetelmiä rakenteen oppimiseen. He jakavat ongelman ensinnäkin kahteen osaan: \emph{edeltäjäjoukon tunnistaminen} (\texttt{parent set identification}) ja \emph{verkon rakenteen optimointi} (\texttt{structure optimization}). 

\citet{scanagatta_survey_2019} toteavat, että edeltäjäjoukon tunnistamisessa naiivi ratkaisu on laskea kaikki mahdolliset Bayes-verkot ja sovittaa niitä. Tällöin aikavaativuudeksi saadaan $O(n^k)$, jossa $n$ on solmujen eli satunnaismuuttujien määrä ja $k$ on yläraja sille, montako vanhempaa solmulla voi olla. Katsauksessa esitellään myös muutamia vaihtoehtoja, joissa muodostettavien Bayes-verkkojen määrää tai niiden hakua rajoittamalla saadaan suhteellisen tehokkaita laskentamentelmiä. Joko luvulle $k$ voidaan asettaa rajoite tai sitten rajoitetta voidaan kiertää esim. käyttämällä ahnetta hakua tai approksimointimenetelmiä.

\citet{scanagatta_survey_2019} mukaan Bayes-verkon rakenteen oppimisen algoritmit keskittyvät useimmiten rakenteen optimoimiseen. Tätä ongelmaa voidaan edelleen lähestyä kahdella eri tavalla:  rajoittamalla tuloksena saatavan Bayes-verkon puuleveyttä tai optimoimalla rakennetta yleisesti eli ilman rajoitteita tuloksena saatavalle Bayes-verkolle. Puuleveyden rajaamiseen perustuvat menetelmät puolestaan perustuvat kaarien heuristiseen lisäämiseen, dynaamiseen ohjelmointiin, kokonaislukuohjelmointiin tai puuleveydeltään $k$ olevien maksimaalisten graafien eli $k$-puiden otoksiin. Yleisen optimoinnin algoritmeille on esitetty useita eri lähestymistapoja: dynaaminen ohjelmointi, lyhimmän polun löytämisen soveltaminen ja branch and bound. Parhaita tuloksia näytettäisiin saavan algoritmeilla, jotka hyödyntävät \emph{kokonaislukujen lineaarista ohjelmointia} (\texttt{integer linear programming, ILP}). \citet{bartlett_integer_2017} käsittelevät artikkelissaan tarkemmin  ILP:n käyttömahdollisuuksia.

\subsection{Pisteyttämiseen perustuvat algoritmit}
\subsection{Rajoitepohjaiset algoritmit}