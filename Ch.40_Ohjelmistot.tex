\chapter{Ohjelmistokirjastoja rakenteen oppimiseen\label{software}}
\citet{scanagatta_survey_2019} esittävät katsauksessaan kattavan yhteenvedon erilaisista ohjelmistoista Bayes-verkkojen käsittelmiseen ja rakenteen oppimiseen. Heidän esittelynsä kattaa niin kaupallisia, kuin vapaasti saatavilla olevia ohjelmistoja ja ohjemistokirjastoja. Tässä luvussa esittelen ja vertailen kolmea hyvin dokumentoitua, vapaasti saatavilla olevaa ja aktiivisesti kehityksessä olevaa ohjelmistokirjastoa suosituimmillelle data-analyyseissä käytetyille ohjelmointikielille. Yhteenveto kirjastoista on esitetty taulukossa ~\ref{table:softat}. \citep{atienza_pybnesian_2022, bayesnetsjl_2021} 

\begin{table}[ht]
\centering
\caption{Ohjelmistokirjastoja Bayes-verkkojen käsittelemiseen\label{table:softat}}
\begin{tabular}{l|| c c c c c} 
kirjasto & kieli & versio & julkaistu & päivitetty & lisenssi \\
\hline \hline 
\href{https://www.bnlearn.com/}{bnlearn} & R & 4.8.1 & & 2022-09-21 & CC BY-SA 3.0 \\
\href{https://github.com/davenza/PyBNesian}{PyBNesian} & Python & 0.4.2 & & 2022-03-26 & MIT \\
\href{https://github.com/sisl/BayesNets.jl}{BayesNets.jl} & Julia & 3.4.0 & & 2022-07-06 & MIT  \\
\hline
\end{tabular}
\end{table}
\section{bnlearn}
Tässä esitellyistä kirjastoista pisin kehityshistoria on ohjelmointikieli R:lle toteutettu kirjasto bnlearn. Kirjaston käyttöä helpottaa myös se, että sen tekijä on julkaissut kaksi kirjaa, joissa kuvataan Bayes-verkkojen käyttämistä juuri tällä kirjastolla, kts. \citep{nagarajan_bayesian_2013, R_bayesian_2014}.

\section{PyBNesian}
Python on hyvin laajalti käytössä data-analytiikassa <viite>. Valitsin Pythonille tehdyistä kirjastoista vertailuun PyBnesian -kirjaston, koska se kattaa hyvin 

\section{BayesNets.jl}


