% \begin{abstract}{finnish}

% Tämä dokumentti on tarkoitettu Helsingin yliopiston tietojenkäsittelytieteen osaston opin\-näyt\-teiden ja harjoitustöiden ulkoasun ohjeeksi ja mallipohjaksi. Ohje soveltuu kanditutkielmiin, ohjelmistotuotantoprojekteihin, seminaareihin ja maisterintutkielmiin. Tämän ohjeen lisäksi on seurattava niitä ohjeita, jotka opastavat valitsemaan kuhunkin osioon tieteellisesti kiinnostavaa, syvällisesti pohdittua sisältöä.


% Työn aihe luokitellaan  
% ACM Computing Classification System (CCS) mukaisesti, 
% ks.\ \url{https://dl.acm.org/ccs}. 
% Käytä muutamaa termipolkua (1--3), jotka alkavat juuritermistä ja joissa polun tarkentuvat luokat erotetaan toisistaan oikealle osoittavalla nuolella.

% \end{abstract}

\begin{otherlanguage}{finnish}
\begin{abstract}
Graafisiin malleihin kuuluva Bayes-verkko kuvaa satunnaismuuttujajoukon yhteistodennäköisyysjakaumaa ja siihen kuuluvien muuttujien välisiä riippuvuussuhteita yhdistelemällä tilastotiedettä, todennäköisyysteoriaa ja graafiteoriaa. Bayes-verkon rakentamisessa on kaksi mielenkiintoista ongelmaa: rakenteen oppiminen ja verkon parametrien laskeminen. Tässä tutkielmassa tehdään katsaus rakenteen oppimisen algoritmeihin ja muutamaan niitä käyttäviin ohjelmointikirjastoihin.

Tämän jälkeen tarkastellaan Bayes-verkkojen sovelluskohteita ja erityistapauksena niiden soveltamismahdollisuuksia työeläkedatassa yksilötason työkyvyttömyysriskin mallintamisessa ja ennustamisessa. Työkyvyttömyysriskin mallintamisella ja ennustamisella on pitkät tilastotieteelliset perinteet, mutta koneoppimismallien soveltaminen niihin on vasta alussa ja julkaistua tutkimusta aiheesta on vähän. Yhteistä julkaistuille ja alustaville koneoppimismallien sovelluksille on, ettei niissä ole tarkasteltu muuttujien välisiä riippuvuussuhteita. Bayes-verkon rakenteen oppiminen tarjoaa mahdollisuuden kuvata näitä riippuvuussuhteita.

\end{abstract}
\end{otherlanguage}
