% \begin{abstract}{finnish}

% Tämä dokumentti on tarkoitettu Helsingin yliopiston tietojenkäsittelytieteen osaston opin\-näyt\-teiden ja harjoitustöiden ulkoasun ohjeeksi ja mallipohjaksi. Ohje soveltuu kanditutkielmiin, ohjelmistotuotantoprojekteihin, seminaareihin ja maisterintutkielmiin. Tämän ohjeen lisäksi on seurattava niitä ohjeita, jotka opastavat valitsemaan kuhunkin osioon tieteellisesti kiinnostavaa, syvällisesti pohdittua sisältöä.


% Työn aihe luokitellaan  
% ACM Computing Classification System (CCS) mukaisesti, 
% ks.\ \url{https://dl.acm.org/ccs}. 
% Käytä muutamaa termipolkua (1--3), jotka alkavat juuritermistä ja joissa polun tarkentuvat luokat erotetaan toisistaan oikealle osoittavalla nuolella.

% \end{abstract}

\begin{otherlanguage}{finnish}
\begin{abstract}
Tilastotiedettä, todennäköisyysteoriaa ja graafiteoriaa yhdisteleviin graafisiin malleihin kuuluva Bayes-verkko on suunnattu syklitön verkko, joka kuvaa satunnaismuuttujajoukon yhteistodennäköisyysjakaumaa ja siihen kuuluvien muuttujien välisiä riippuvuussuhteita. Bayes-verkkoja käyttävien mallien rakentamisessa on kaksi mielenkiintoista ongelmaa: verkon rakenteen oppiminen ja verkon parametrien laskeminen. Tässä tutkielmassa tehdään katsaus Bayes-verkon rakenteen oppimisen menetelmiin.

Bayes-verkoilla on sovelluskohteita usealla eri tieteenalalla. Tässä tutkielmassa tarkastellaan niiden soveltamismahdollisuuksia työkyvyttömyysriskin mallintamisessa ja ennustamisessa. Työkyvyttömyysriskin mallintamisella ja ennustamisella on pitkät tilastotieteelliset perinteet, mutta koneoppimismallien soveltamisessa ollaan vasta alussa ja julkaistua tutkimusta aiheesta on vähän. Yhteistä niin julkaistuille kuin alustaville koneoppimismallien sovelluksille työkyvyttömyysriskien parissa on, ettei niissä ole tarkasteltu muuttujien välisiä riippuvuussuhteita kuin korkeintaan pareittain. 

Tutkielmassa esitetään, että Bayes-verkon rakenteen oppimisen menetelmät tarjoavat mahdollisuuden kuvata näitä riippuvuussuhteita, ja sitä kautta lisätä ymmärrystä yksilötason työkyvyttömyysriskiin vaikuttavista muuttujista ja niiden yhteisvaikutuksista. Parhaimmillaan Bayes-verkkoja voitaisiin käyttää jopa työkyvyttömyysriskiä alentavien interventioiden, kuten ammatillisen kuntoutuksen kohdentamiseen ja kehittämiseen.

\end{abstract}
\end{otherlanguage}
