\chapter{Sovelluskohteet\label{applications}}
\section{Bayes-verkon rakenteen oppimisen sovelluksia}
Voidaan ajatella, että on neljä sovelluskohdetta:
1) ennuste
2) ennuste 2 -> taustamuuttujan ennustaminen kun "lopputulos" tiedetään
3) graafin tulkinta -> muuttujien interaktiot
4) interventio -> mielenkiintoinen, mutta vaatisi kausaliteetin tuntemista

1) graafin tulkinta
2) ennuste
3) tiedetään että jää eläkkeelle -> voidaan ennustaa taustamuuttuja (ammatillinen kuntoutus voisi olla mielenkiintoinen -> onko jokin demografinen ryhmä, jolle alitarjotaan kuntoutusta)
4) interventiot: jos olisi kausaalinen verkko, voitaisiin löytää kohta jota olisi pitänyt muuttaa --> ammatillinen kuntoutus? -> kenelle pitäisi tarjota enemmän, kenellä jää käyttämättä koik?


Insinööritieteessä vikadiagnostiikka ja luotettavuusarviointi \citep{zhang_brief_2019}, lääketieteellinen diagnostiikka esim. \citep{mittal_review_2011}, anomalisten tapahtumien tunnistaminen \citep{kaur_review_2013} ja vakuutusalalla \citep{ramsahai_connecting_2020}. <kts. artikkelit, jotka viittaavat scutarin ym. katsaukseen jos tarvitaan pidempää listaa)


\section{Mahdollisia sovelluskohteita työeläkedatassa}
Eläkedatassa mielenkiintoisia kysymyksiä ja pohdintoja \citep{gross_machine_2020}.

Yksilö- ja ryhmätason eläköitymisriskien ymmärtäminen. Taustamuuttujat: henkilötiedot, hakemuksessa mainittu diagnoosi, ammatillinen kuntoutus, ammattinimike ja toimiala, tulotaso, asuinalueen sosioekonomiset tekijät

Triviaalisti ikä ennustaa eläköitymistä. Lisäksi olemme esimerkiksi havainneet, että tehtävänimike on yhteydessä eläköitymisriskiin / 1000 työntekijää. Mielenterveyseläkkeissä selittäviksi tekijöiksi diagnoosin lisäksi palkattomat jaksot ja asuinalueeseen liittyvä sosioekonominen tieto, tätä on osoittanut myös \citep{karolaakso_contextual_2021, karolaakso_socioeconomic_2020}.