\chapter{Koneoppiminen työkyvyttömyysriskin mallintamisessa\label{applications}}

Väestön vanhetessa työvoiman saatavuuden varmistaminen ja työurien pidentäminen on noussut tärkeäksi tavoitteeksi työelämän kehittämisessä \citep{noauthor_paaministeri_2019}. Työkyvyttömyysriskin hallinta on toimintaa, jolla pyritään vähentämään työntekijöiden riskiä päätyä työkyvyttömäksi ja joutua työkyvyttömyysetuudelle. Ennenaikainen työkyvyttömyys ja työkyvyttömyyseläkkeet muodostavat ison kuluerän niin työeläkejärjestelmälle, kuin koko yhteiskunnalle. Suomessa työkyvyttömyysriskin hallinnasta vastaavat työeläkeyhtiöt.

Työeläkeyhtiölle kertyvä data sisältää vakuutetun työntekijän henkilötiedon lisäksi tietoa hänen työhistoriastaan ja ansioistaan. Lisäksi työkyvyttömyysetuuksiin (työkyvyttömyyseläke tai ammatillinen kuntoutus) liittyvien hakemusten myötä kertyy tietoa työkyvyttömyyden taustalla olevasta sairaudesta, viasta tai vammasta. Tätä dataa voidaan käyttää sen mallintamiseen, minkälaiset riskitekijät ennustavat työkyvyttömyysetuudelle joutumista. On myös mahdollista laskea riski sille, että henkilö- ja taustatietojen perusteella jaoteltuun ryhmään kuuluva henkilö hakee etuutta tai tarkastella todennäköisyyttä, jolla haettu etuus myönnetään, kun henkilö- ja taustatiedot tunnetaan. Lisäksi dataa on mahdollista rikastaa liittämälle siihen esimerkiksi hakijan asuinalueeseen liittyviä sosioekonomisia tietoja.

Työkyvyttömyysriskiä ja työkyvyttömyysetuuden hakualttiutta on hiljattain arvioitu suomalaisissa alan tutkimuksissa. \citet{karolaakso_contextual_2021, karolaakso_socioeconomic_2020} ovat osoittaneet, että mielenterveysperusteisia työkyvyttömyyseläkkeitä selittävät osaltaan hakijan asuinalueeseen sosioekonomiset indikaattorit, kuten työllisyysaste ja mielenterveyspalveluiden käyttöaste. \citet{perhoniemi_determinants_2020, perhoniemi_tyokyvyttomyyselakehakemusten_2020} ovat osoittaneet, että eläkkeen hakijan sosioekonominen asema ja aiempi työttömyyshistoria on yhteydessä siihen, myönnetäänkö hakijalle työkyvyttömyyseläke, ja tämä yhteys selittyy heidän mukaansa osittain sillä, että matalamassa sosioekonomisessa asemassa olevat hakevat suhteellisesti useammin työkyvyttömyyseläkettä, kuin korkeammassa sosioekonomisessa asemassa olevat. \citet{bethge_using_2021} puolestaan osoittivat saksalaisella aineistolla sekä pitkien että lyhyiden työttömyys- ja sairauspoissaolojaksojen olevan yhteydessä korkeampaan eläköitymisriskiin. Huomionarvoista oli, että näiden kahden muuttujan yhteisvaikutus lisää tätä riskiä.

\citet{gross_machine_2020} esittelevät esipuheessaan \emph{Journal of Occupational Rehabilitation}-julkaisun teemanumeron, joka käsittelee koneoppimisen soveltamista työkyvyttömyysriskiin. Heidän mukaansa työkyvyttömyysriskin ennustamisessa on perinteisesti luotettu tilastollisiin menetelmiin, etenkin regressioanalyysiin ja koneoppimisen sovelluksissa ollaan vielä varsin alkutekijöissä. Tämän tutkielman kirjoittamishetkellä en löytänyt kuvausta siitä, että nimenomaan Bayes-verkkojen hyödyntämistä olisi tutkittu työkyvyttömyysriskin mallintamisessa. 

Suomessa koneoppimismallien hyödyntämisestä on tehty muutamia alustavia tutkimuksia. Eläketurvakeskus on julkaissut alustavia tietoja omista kokeiluistaan koneoppimismenetelmistä työkyvyttömyysriskin ennustamisessa. \citet{varis_aketurvakeskuksen_2018} kuvailee näitä alustavia tuloksia neljällä eri luokittelumenetelmällä: regressio, päätöspuu, gradient boosting \citep{friedman_greedy_2001} ja neuroverkko, ja toteaa parhaiten ennustavien muuttujien noudattavan aiempaa tutkimustietoa: sosioekonomiset tekijät, henkilön ansiohistoria ja aiemmat etuudet ennustavat hyvin erityisesti korkeaa työkyvyttömyyden riskiä. \citet{sami_tyokyvyttomyyselakeratkaisun_2022} esittelee vakuutusmatematiikan suppeassa SVH-lopputyössään alustavia tuloksia erilaisten koneoppimismallien mahdollisuuksista henkilötason työkyvyttömyysriskin ennustamisesta eläkeyhtiön aineistossa.

\citet{saarela_work_2022} vertailivat työkyvyttömyysriskin tunnistamiseen kahta lähestymistapaa. Ensimmäisessä $M_{pension}$ mallissa tutkittiin \citet{varis_aketurvakeskuksen_2018} esittelemiä tuloksia ja toisessa $M_{health}$ mallissa \citet{huhta-koivisto_work_2020} diplomityössään esittelemää luonnollisen kielen tunnistamista työterveyden potilastietoihin soveltavaa mallia. \citet{saarela_work_2022} mukaan molempien mallien ennustekyky oli varsin hyvä, mutta eläkerekisteridataa käyttävä $M_{pension}$ osumatarkkuus oli parempi. Tässä kannattaa huomioida, että henkilön jättäessä työkyvyttömyysetuushakemuksen, on hänen terveydentilansa ja työkykynsä jo selkeästi alentunut. Tästä seuraa, että hakijan työkyvyttömyyden riski on suurempi kuin sellaisella, joka ei ole etuutta hakenut. 

\citet{na_machine_nodate} tutkivat työhön palaamisen ennustetta sairauslomalta gradient-boostingin avulla käyttäen laajaa taustamuuttujajoukkoa ja osoittivat, että malli ennustaa varsin hyvin yleisellä tasolla työhön paluuta. Sosioekonomisten muuttujien osalta tulokset olivat hyvin samansuuntaisia kuin \citet{karolaakso_contextual_2021, karolaakso_socioeconomic_2020} tutkimuksessa esitetyt. 

Graafisia malleja soveltaneet \citet{airaksinen_development_2017} kehittivät suomalaisella aineistolla työkyvyttömyyden yksilöriskin ennustemallin kahdeksalle muuttujalle. He käyttivät mallin rakentamisessa BIC:lle sukua olevaa \emph{AIC}-pisteytystä (\texttt{Akaike Information Criterion}). Lisäksi he tutkivat käytettyjen muuttujien riippuvuussuhteita muuttujapareittain. AIC:n ja BIC:n yhtäläisyyksiä ja eroja ovat vertailleet esim. \citet{ding_model_2018}.

Kaikille tässä kuvailluille koneoppimisen sovelluksille työkyvyttömyysriskin ennustamisessa on yhteistä se, ettei muuttujien yhteisvaikutuksia tai riippuvuuksia ole tarkasteltu kuin korkeintaan pareittain vertailemalla. Tämä johtunee tilastollisen mallintamisen perinteestä, jossa on ennemmin pyritty välttämään kuin ymmärtämään mallin selittävien muuttujien laajoja yhteisvaikutuksia. Vaikuttaa kuitenkin hyvin todennäköiseltä, että työkyvyttömyysriskiin vaikuttavilla muuttujien välillä voisi olla riippuvuuksia. Esimerkiksi henkilön työttömyyshistoriaa, ansiotasoa ja hänen asuinalueensa työllisyystilannetta on vaikea ajatella riippumattomina muuttujina.



