\chapter{Sovelluskohteet\label{applications}}
Bayes-verkoille on löydettty sovelluskohteitea monilla tieteenaloillaa. Insinööritieteessä niitä voidaan käyttää vikadiagnostiikkaan ja luotettavuuden arviointiin \citep{zhang_brief_2019}, lääketietessä diagnostiikan apuna \citep{mittal_review_2011}, tietoverkoissa poikkeavien tapahtumien ja mahdollisten tunkteutujien tunnistamiseen \citep{kaur_review_2013}. Graafisten mallien sovelluskohteita vakuutusalalla ovat mm. tapahtumien frekvenssin ja vakavuuden välisen riippuvuuden tarkastelun, riippumattomien tekijöiden taustalla vaikuttavien välillisten muuttujien tunnistamisen lineaarissa malleissa ja latenttien muuttujien tunnistamisen puutteellisessa datassa \citep{ramsahai_connecting_2020}. 

Yleisesti ottaen Bayes-verkon rakenteen oppimisella voidaan ajatella olevan neljä eri sovelluskohdetta: 1) posterioritapahtuman ennustaminen, 2) prioritapahtumien ennustaminen kun posterioritapahtuma on tunnettu, 3) graafin tulkitseminen muuttujien välsiten yhteyksien ymmärtämiseksi ja 4) intervention kohdentaminen sopivaan kohtaan. On huomioitava, että intervention kohdentaminen on merkittävästi muita sovelluskohteita haastavampi, koska se edellyttää tietoa muuttujien välisten riippuvuussuhteiden kausaalisuudesta.

\section{Työkyvyttömyysriskin mallintaminen ja ennustaminen}

Väestön vanhetessa työvoiman saatavuuden varmentistaminen ja työurien pidentäminen on noussut tärkeäksi tavoitteeksi työelämän kehittämisessä. Työkyvyttömyysriskin hallinta on toimintaa, jolla pyritään vähentämään työntekijöiden riskiä päätyä työkyvyttömäksi ja joutua työkyvyttömyysetuudelle. Ennenaikainen työkyvyttömyys ja työkyvyttömyyseläkkeet muodostavat ison kuluerän niin työeläkejärjestelmälle, kuin koko yhteiskunnalle. Suomessa työkyvyttömyysriskin hallinnasta vastaavat työeläkeyhtiöt.

Työeläkeyhtiölle kertyvä data sisältää vakuutetun työntekijän henkilötiedon lisäksi tietoa hänen työhistoriastaan ja ansioistaan. Lisäksi työkyvyttömyysetuuksiin (työkyvyttömyyseläke tai ammatillinen kuntoutus) liittyvien hakemusten myötä kertyy tietoa työkyvyttömyyden taustalla olevasta sairaudesta tai vammasta. Tämän datan perusteella voidaan arvioida minkälaiset riskitekijät ennustavat työkyvyttömyysetuudelle joutumista. On myös mahdollista laskea riski sille, että henkilö- ja taustatietojen perusteella jaoteltuun ryhmään kuuluva henkilö hakee etuutta tai tarkastella todennäköisyyttä, jolla haettu etuus myönnetään kun henkilö- ja taustatiedot tunnetaan. Työkyvyttömyysriskiä ja työkyvyttömyysetuuden hakualttiutta on hiljattain arvioitu suomalaisissa alan tutkimuksissa. \citet{karolaakso_contextual_2021, karolaakso_socioeconomic_2020} ovat osoittaneet, että mielenterveysperustesia työkyvyttömyyseeläkkeitä selittävät osaltaan asuinalueeseen liittyvä sosioekonomiset indikaattorit. \citet{perhoniemi_determinants_2020, perhoniemi_tyokyvyttomyyselakehakemusten_2020} ovat osoittaneet, että sosioekonominen asema ja aiempi työttömyyshistoria on yhteydessä siihen, myönnetäänkö hakijalle työkyvyttömyyseläke kun hän sitä hakee. Tämä yhteys selittyy osittain sillä, että matalamassa sosioekonomisessa asemassa olevat hakevat suhteellisesti useammin työkyvyttömyyseläkettä, kuin korkeammassa sosioekonomisessa asemassa olevat.

\citet{gross_machine_2020} esittelevät esipuheessaan lehden teemanumeron, joka käsittelee koneoppimisen soveltamista työkyvyttömyysriskiin. Heidän mukaansa yökyvyttömyysriskin ennustamisessa on perinteisesti luotettu tilastollisiin menetelmiin, etenkin regressioanalyysiin ja koneoppimisen sovelluksissa ollaan vielä varsin alkutekijöissä. En ole löytänyt kuvausta siitä, että Bayes-verkkojen hyödyntämistä olisi tutkittu työkyvyttömyysriskin mallintamisessa. 

 Suomessa Eläketurvakeskus on julkaissut alustavia tietoja omista kokeiluistaan koneoppimismenetelmistä työkyvyttömyysriskin ennustamisessa. \citet{varis_aketurvakeskuksen_2018} kuvailee alustavia tuloksia neljällä eri luokittelumenetelmällä: regressio, päätöspuu, gradient boosting (kts. \citet{friedman_greedy_2001})  ja neuroverkko, ja toteaa, että parhaiten ennustavat muuttujat noudattavat aiempaa tutkimustietoa: sosioekonomiset tekijät, henkilön ansiohistoria ja aiemmat etuudet ennustavat hyvin erityisesti korkeaa työkvyttömyyden riskiä.  \citet{saarela_work_2022} vertailivat työkyvyttömyysriskin tunnistamiseen kahta lähestymistapaa. Ensimmäisessä $M_{pension}$ mallissa tutkittiin \citet{varis_aketurvakeskuksen_2018} esittelemiä tuloksia ja toisessa $M_{health}$ mallissa \citet{huhta-koivisto_work_2020} diplomityössään esittelemää luonnollisen kielen tunnistamista työterveyden potilastietoihin soveltavaa mallia. \citet{saarela_work_2022} mukaan molempien mallien ennustekyky oli varsin hyvä, eläkerekisteridataaa käyttävä $M_{pension}$ osumatarkkuus oli parempi. Toki tässä kannattaa huomioida, että henkilön jättäessä työkyvyttömyysetuushakemuksen on hänen terveydentilansa ja työkykynsä jo selkeästi alentunut eli työkyvyttömyyden riski on suurempi kuin sellaisella, joka ei ole etuutta hakenut. Graafisia malleja soveltaneet \citet{airaksinen_development_2017} kehittivät suomalaisella aineistolla työkyvyttömyyden yksilöriskin ennustemallin kahdeksalle muuttujalle. He käyttivät mallin rakentamisessa BIC:lle sukua olevaa \emph{AIC}-pisteytystä (\texttt{Akaike Information Criterion}) ja lisäksi tutkivat käytettyjen muuttujien kahdenvälisiä riippuvuussuhteita. AIC:n ja BIC:n yhtäläisyyksiä ja eroja ovat vertailleet esim. \citet{ding_model_2018}.

Kaikille tässä kuvailemilleni koneoppimisen sovelluksille työkyvyttömyysriskin ennustamisessa on yhteistä se, ettei muuttujien yhteisvaikutuksia tai riippuvuuksia ole tarkasteltu muutoin kuin korkeintaan parittain vertailemalla. Bayes-verkon rakenteen oppiminen tarjoaa menetelmänä erinomaisen mahdollisuuden hahmottaa näitä mahdollisia riippuvuuksia, kun meillä on jo tiedossa muuttujajoukko, jonka tiedämme toimivan hyvin työkyvyttömyysriskin ennustamiseen. Suunnattuja verkkoja on hyödynnetty riippuvuuksien kuvaamiseen jo työkyvyttöömyysriskille hyvin läheisissä aiheissa, kuten \citet{elovainio_is_2021, elovainio_network_2020} tutkimuksissa masennuksen taustatekijöistä ja niiden keskinäisistä yhteyksistä sekä \citet{xu_workplace_2022} tutkimuksessa työpaikan voimavaratekijöiden suojaavasta vaikutuksesta sydän- ja verisuonitauteihin sairastumisen riskistä <tässä käytettiin BIC, mutta verkon rakenne tuli aiemman tiedon perusteella, ei opittu datasta>. 

Koneoppimismallien yhtenä ongelmana on niiden läpinäkymättömyys <VIITE?>. Malli antaa datan pohjalta ennusteen tai luokittelun, mutta voi olla vaikea saada tietoa siitä miten malliin syötetyt muuttujat ovat vaikuttaneet lopputulokseen. Malleissa käytettävien muuttujien painoarvoja voidaan tarkastella jälkikäteen peliteoriasta lähtöisin olevien \emph{Shapley-arvojen} avulla. Shapley-arvojen tarkempi käsittely rajautuu tämän tutkielman ulkopuolelle, mutta niiden käyttämisestä koneoppimismallien arvioimiseen löytyy lisää tietoa esim. \citet{merrick_explanation_2020}. Shapley-arvot eivät kuitenkaan suoraan anna tulkintamahdollisuutta muuttujien välisistä riippuvuuksista, mutta niiden avulla voitaisiin valita Bayes-verkkoon mallissa voimakkaimmin vaikuttavat muuttujat. Toinen vaihtoehto olisi mahdollisten riippuvuuksien tai riippuvuusketjujen tunnistaminen niiden muuttujien taustalla, jotka saavat korkeimmat Shapley-arvot koneoppimismallissa.


%%Mikä sitten olisi Bayes-verkon sovelluskohde?

%%1) graafin tulkinta
%%2) ennuste
%%3) tiedetään että jää eläkkeelle -> voidaan ennustaa taustamuuttuja (ammatillinen kuntoutus voisi olla mielenkiintoinen -> onko jokin demografinen ryhmä, jolle alitarjotaan kuntoutusta)
%%4) interventiot: jos olisi kausaalinen verkko, voitaisiin löytää kohta jota olisi pitänyt muuttaa --> ammatillinen kuntoutus? -> kenelle pitäisi tarjota enemmän, kenellä jää käyttämättä koik?
