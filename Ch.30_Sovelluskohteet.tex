\chapter{Sovelluskohteet\label{applications}}
\section{Bayes-verkon rakenteen oppimisen sovelluksia}
Voidaan ajatella, että on neljä sovelluskohdetta:
1) ennuste
2) ennuste 2 -> taustamuuttujan ennustaminen kun "lopputulos" tiedetään
3) graafin tulkinta -> muuttujien interaktiot
4) interventio -> mielenkiintoinen, mutta vaatisi kausaliteetin tuntemista


Insinööritieteessä vikadiagnostiikka ja luotettavuusarviointi \citep{zhang_brief_2019}, lääketieteellinen diagnostiikka esim. \citep{mittal_review_2011}, anomalisten tapahtumien tunnistaminen \citep{kaur_review_2013} ja vakuutusalalla \citep{ramsahai_connecting_2020}. <kts. artikkelit, jotka viittaavat scanagatta ym. katsaukseen jos tarvitaan pidempää listaa)

\section{Mahdollisia sovelluskohteita työeläkedatassa}
Työeläkeyhtiölle kertyvä data sisältää vakuutettujen työntekijöiden henkilötiedon lisäksi tietoa hänen työhistoriastaan ja ansioistaan. Lisäksi työkyvyttömyysetuuksiin (työkyvyttömyyseläke tai ammatillinen kuntoutus) liittyvien hakekmusten myötä kertyy mm. tietoa työkyvyttömyyden taustalla olevasta sairaudesta tai vammasta. Tämän datan perusteella voidaan arvioida minkälaiset riskitekijät ennustavat työkyvyttömyysetuudelle joutumista. On myös mahdollista laskea riski sille, että henkilö hakee etuutta. Työkyvyttömyysetuuden hakualttiutta ja eläkeriskiä onkin arvioitu suomalaisissa alan tutkimuksissa. Esim. mielenterveysperustesia eläkkeitä selittäviksi on osoitettu asuinalueeseen liittyvä sosioekonominen tieto \citep{karolaakso_contextual_2021, karolaakso_socioeconomic_2020}. Laaksonen ym. osoittaneet, että sosioekonominen asema on yhteydessä eläkkeen hakualttiuteen <tarkista tämä>.

Työkyvyttömyysriskin ennustamisessa on \citet{gross_machine_2020} mukaan perinteisesti luotettu varsin vahvasti tilastollisiin menetelmiin, ja etenkin regressioanalyysiin. Omassa työpaikassani olemme kokeilleet regression lisäksi erityisesti \citet{friedman_greedy_2001} esittelemiä \emph{gradient boosting} -menetelmällä tehostettuja päätöksentekopuita. Olemme myös tarkastelleet nihin malleihin mukaan otettujen muuttujien painoarvoja peliteoriasta lähtöisin olevien \emph{Shapley-arvojen} avulla. Näiden tarkempi käsittely rajautuu tämän tutkielman ulkopuolelle, mutta niistä löytyy lisää tietoa esim. \citet{merrick_explanation_2020}. Shapley-arvojen avulla voidaan päätellä mitkä muuttujat ovat vaikuttaneet eniten mallin tekemään ennusteeseen, mutta näiden muuttujien yhtisvaikutusten tutkiminen on hankalampaa. Kuvaajien avulla olemme pystyneet tarkastelemaan 2-3 muuttujan mahdollisia yhteisvaikutuksia silmämääräisesti.

Kun meillä nyt on tiedossa sellainen muuttujajoukko, jonka tiedämme toimivan hyvin työkyvyttömyysriskin ennustamiseen, voisi Bayes-verkko menetelmänä tarjota mahdollisuuden sekä kuvata, että laskea ennustemalleihin voimakkaasti vaikuttavien muuttujien välisiä riippuvuuksia. Edellisessä luvussa esittelemäni mallin \ref{a} hengessä Bayes-verkot voisivat tarjota mahdollisuuksia.

Tämäntyyppistä tutkimusta on jo tehty väestötasolla \citet{airaksinen_development_2017} kehittivät työkyvyttömyyden yksilöriskin ennustamllin kahdeksalle muuttujalle. Mallin rakentamisessa he käyttivät AIC ja tutkivat käytettyjen muuttujien kahdenvälisiä riippuvuussuhteita. Tästä olisi enää pieni hyppäys siihen, että kaikki muuttujien väliset riippuvuudet voitaisiin tehdä näkyviksi Bayes-verkon rakenteen oppimisen algoritmeilla.

Mikä sitten olisi Bayes-verkon sovelluskohde?

1) graafin tulkinta
2) ennuste
3) tiedetään että jää eläkkeelle -> voidaan ennustaa taustamuuttuja (ammatillinen kuntoutus voisi olla mielenkiintoinen -> onko jokin demografinen ryhmä, jolle alitarjotaan kuntoutusta)
4) interventiot: jos olisi kausaalinen verkko, voitaisiin löytää kohta jota olisi pitänyt muuttaa --> ammatillinen kuntoutus? -> kenelle pitäisi tarjota enemmän, kenellä jää käyttämättä koik?
