\chapter{Johdanto\label{intro}} 

Väestön vanhetessa työvoiman saatavuuden varmentaminen ja työurien pidentäminen on noussut tärkeäksi tavoitteeksi työelämän kehittämisessä \citep{noauthor_paaministeri_2019}. Työkyvyttömyysriskin hallinta on laaja-alaista toimintaa, jolla pyritään vähentämään työntekijöiden riskiä päätyä työkyvyttömäksi ja joutua työkyvyttömyyseläkkeelle. 

Ennenaikainen työkyvyttömyys ja työkyvyttömyyseläkkeet muodostavat ison kuluerän niin työeläkejärjestelmälle eläkemaksuina, kuin koko yhteiskunnalle menetettynä työpanoksena. Vuonna 2021 työkyvyttömyyseläkkeiden suorat kustannukset eläkemaksuina olivat $2.4$ miljardia euroa \citep{etk_show_me_the_money}. Työkyvyttömyyden kustannukset menetettynä työpanoksena ovat huomattavasti suuremmat. \citet{rissanen_menetetyn_2014} laskelman mukaan vuonna 2012 ne ylsivät $8$ miljardiin euroon. Riskitekijöiden ja niiden välisten yhteyksien ymmärtämisellä on siis suuri kansantaloudellinen vaikutus. 

Työkyvyttömyysriskin mallintamisessa ja ennustamisessa on käytetty tilastollisia menetelmiä \citep{gross_machine_2020}, mutta hiljattain mallintamiseen on kokeiltu myös erilaisia koneoppimismalleja. Näissä tilasto-, ja koneoppimismalleissa muuttujien välisiä riippuvuuksia ei ole kuitenkaan julkaistuissa tutkimuksissa tarkasteltu kuin korkeintaan pareittain. 

Muuttujien välisiä riippuvuuksia voidaan tarkastella yleisellä tasolla graafisten mallien avulla. \citet{ramsahai_connecting_2020} esittelee graafisten mallien sovelluskohteita vakuutusalalla. Yhtenä sovelluskohteena hän mainitsee riskejä kuvaavien muuttujien välisten yhteyden kuvaamisen ja ymmärtämisen. Tässä tutkielmassa esitän, että nimenomaan graafisiin malleihin kuuluvan Bayes-verkon ja sen rakenteen oppimiseen tarkoitettujen menetelmien avulla tällaisia riippuvuuksia voitaisiin kuvata ja laskea työeläkedatasta. Tämä mahdollistaa paremman ymmärryksen saavuttamisen työkyvyttömyysriskiin vaikuttavien muuttujien välisistä riippuvuuksista. 

Luvussa \ref{background} esittelen niitä matemaattisia käsitteitä, joita tarvitsemme luvussa \ref{bayes} esittelemääni Bayes-verkon käsitteen ja sen rakenteen oppimiseen käytettävien algoritmien perusteiden ymmärtämiseen. Koska tutkielmani yhtenä tavoitteena on käytännön sovelluskohteiden löytäminen, esittelen luvussa \ref{bayes} myös muutaman avoimen lähdekoodin ohjelmistokirjaston Bayes-verkkojen käsittelemiseen. Luvussa \ref{applications} tarkastelen, miten koneoppimismalleja on käytetty työkyvyttömyysriskin mallintamisessa ja ennustamisessa, ja luvussa \ref{conclusions} pohdin miten Bayes-verkkoja voisi soveltaa tähän mallintamiseen. 