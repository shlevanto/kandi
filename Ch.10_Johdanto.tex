\chapter{Johdanto\label{intro}}

\section{Johdanto}
Tarkastelen tässä tutkielmassa graafisiin malleihin kuuluvien Bayes-verkkojen rakenteen oppimisen käyttöä vakuutusdatan analysoimiseen. \citet{ramsahai_connecting_2020} kuvailee muutamia tapoja soveltaa graafisia malleja vakuutusdatassa ja oman tutkielmani kannalta mielenkiintoisin esimerkki liittyy vähemmän ilmeisten tai latenttien riippuvuuksien tunnistamiseen eri satunnaismuuttujien välillä. Erityistapauksena minua kiinnostaa Bayes-verkkojen soveltamismahdollisuudet työeläkkeisiin liittyvässä henkilökeskeisessä datassa.

\section{Tausta ja terminologia}
Bayes-verkko on \emph{suunnattu syklitön verkko} (\texttt{Directed Acyclic Graph, DAG}), joka kuvaa satunnaismuuttujien yhteistodennäköisyysjakaumaa \citep{ruggeri_bayesian_2008}. Tarkastellaan seuraavaksi mitä tämä tarkoittaa.

\subsection{Verkko eli graafi}
\emph{Verkko} eli \emph{graafi} on matematiikassa ja tietojenkäsittelytieteessä yleisesti käytössä oleva käsite. Graafien avulla voidaan yleisesti kuvata ja mallintaa monimutkaisia ja toistensa kanssa risteäviä yhteyksiä intuitiivisella ja ymmärrettävällä tavalla. Graafeja ja graafiteoriaa on käsittelee tarkemmin esim. \citet{diestel_graph_2017}.

Matemaattisesti ilmaistuna graafi $G$ on pari, joka koostuu joukosta \emph{solmuja} (\texttt{node} tai \texttt{vertex}) $V$ ja joukosta niitä yhdistäviä \emph{kaaria} (\texttt{edge}) $E$. Kaaret voivat olla suunnattuja, eli niille on erikseen määritelty alku- ja päätesolmu tai ne voivat olla suuntaamattomia. Graafia kutsutaan suunnatuksi tai suuntaamattomaksi sen mukaisesti, minkälaisia kaaria siinä on. Suunnattu graafi määritellään formaalisti 
$$
    G = \{V, E\}, E \subseteq \{(a,b) : a,b \in V \}
$$
ja suuntaamaton graafi vastaavasti
$$
    G = \{V, E\}, E \subseteq \{\{a,b\} : a,b \in V \}
$$

Graafin solmujen $x_0, x_k$ välillä on \emph{polku}, jos solmusta $x_0$ lähteviä kaaria pitkin kulkemalla voidaan päästä solmuun $x_k$. Matemaattisesti ilmaistuna polku $P=(V,E)$ on graafi tai graafin solmujen ja niitä yhdistävien kaarien osajoukko, jossa 
$$
    V = \{x_0, x_1, \ldots x_k \} \qquad E = \{(x_0, x_1), (x_1,x_2),\ldots(x_{k-1}, x_k)\}
$$ 
siten, että jokainen $x_i$ on erillinen solmu. Polulla järjestyksessä ensiksi olevaa solmua kutsutaan \emph{edeltäjäksi} ja seuraavaa solmua sen \emph{jälkeläiseksi}.

Polku on \emph{sykli}, jos siinä on vähintään kolme solmua ja sen alkusolmu on sama kuin päätesolmu eli sykli $C = (V, E)$, on polku jossa
$$
    V = \{x_0, x_1, \ldots x_k \} \qquad E = \{(x_0, x_1), (x_1,x_2),\ldots(x_{k-1}, x_k), (x_k, x_0)\}
$$
Graafi on \emph{syklitön}, kun se tai mikään sen solmujen ja niitä yhdistävien kaarien osajoukko ei muodosta sykliä. 

\subsection{Satunnaismuuttujat ja todennäköisyysjakaumat}
Todennäköisyysteoria on vakiintunut matematiikan osa-alue, joka muodostaa perustan nykyaikaiselle tilastotieteelle. Todennäköisyyslaskennan käsitteistöä ja teoriaa esitellään laajemmin mm. \citet{dasgupta_probability_2011}.

\emph{Satunnaismuuttuja} on funktio, joka kuvaa jonkin tietyn satunnaistapahtuman määräämää lukua. Satunnaismuuttujan arvo siis vaihtelee havaintotapauksesta toiseen jonkin tietyn arvojoukon sisällä. Esimerkiksi yksittäisen nopanheiton tulos on tapahtuma, joka voi saada jonkin arvon joukosta $\{1,2,3,4,5,6\}$. Satunnaismuuttujalla kuvataan tässä tapauksessa todennäköisyyttä saada jokin tietty tulos yksittäisestä nopanheitoista. \emph{Todennäköisyysjakauma} puolestaan kuvaa todennäköisyyttä, jolla satunnaismuuttuja saa jonkin tietyn arvon.

Satunnaismuuttujien \emph{yhteistodennäköisyysjakauma} kuvaa kahden tai useamman muuttujan kaikkien mahdollisten lopputulosten ja niiden todennäköisyyksien joukkoa. Esimerkiksi jos heitetään noppaa kaksi kertaa, voidaan yhteistodennäköisyysjakauman avulla selvittää todennäkäisyys saada ensimmäisellä nopanheitolla pienempi arvo kuin jälkimmäisellä. MuuttujieYhteistodennäköisyysjakauman yksittäisten muuttujien jakaumia kutsutaan \emph{reunajakaumiksi}.

Satunnaistapahtumat voivat olla keskenään \emph{riippuvaisia} tai \emph{riippumattomia}. Tapahtumat $A$ ja $B$ ovat riippumattomia $A \indep B$, jos
$$
    P(A)P(B|A) = P(A)P(B) 
$$
Vastaavasti tapahtumat $A$ ja $B$ ovat riippuvaisia, jos ehdollinen todennäköisyys $P(A|B) \not= P(B)$ eli tapahtuman $A$ on lopputulos vaikuttaa tapahtumaan $B$.

Jos esimerkiksi  kulhossa on sama määrä mustia ja valkoisia palloja ja niistä nostetaan ilman takaisinpanoa kaksi palloa, vaikuttaa ensimmäisen nostetun pallon väri $A$ todennäköisyyteen jolla jälkimmäisen nostetun pallon väri $B$ on valkoinen tai musta. Jos ensimmäinen nostettu pallo pannaan takaisin kulhoon ennen jälkimmäisen pallon nostamista, ovat tapahtumat $A$ ja $B$ riippumattomia. Vastaavasti satunnaismuuttujat $X$ ja $Y$ ovat riippuvaisia jos niiden kuvaamat tapahtumat ovat riippuvaisia. Muuttujien $X$ ja $Y$ yhteistodennäköisyysjakaumassa $X \indep Y$, jos $X(Y|X) ) = XY$.


 

