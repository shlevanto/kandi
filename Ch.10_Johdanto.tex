\chapter{Johdanto\label{intro}}

Väestön vanhetessa työvoiman saatavuuden varmentaminen ja työurien pidentäminen on noussut tärkeäksi tavoitteeksi työelämän kehittämisessä \citep{noauthor_paaministeri_2019}. Työkyvyttömyysriskin hallinta on laaja-alaista toimintaa, jolla pyritään vähentämään työntekijöiden riskiä päätyä työkyvyttömäksi ja joutua työkyvyttömyyseläkkeelle.
Ennenaikainen työkyvyttömyys ja työkyvyttömyyseläkkeet muodostavat ison kuluerän niin työeläkejärjestelmälle eläkemaksuina, kuin koko yhteiskunnalle menetettynä työpanoksena. Vuonna 2021 työkyvyttömyyseläkkeiden suorat kustannukset eläkemaksuina olivat $2.4$ miljardia euroa \citep{etk_show_me_the_money}. Työkyvyttömyyden kustannukset menetettynä työpanoksena ovat huomattavasti laajemmat, \citet{rissanen_menetetyn_2014} laskelman mukaan vuonna ne ylsivät $8$ miljardiin euroon. Riskitekijöiden ja niiden välisten yhteyksien ymmärtämisellä on siis suuri kansantaloudellinen vaikutus.

Työkyvyttömyysriskin mallintamisessa ja ennustamisessa on käytetty tilastollisia mentelmiä \citep{gross_machine_2020}, mutta hiljattain mallintamiseen on kokeiltu myös erilaisia koneoppimismalleja. Näissä tilasto-, ja koneoppimismalleissa muuttujien välisiä riippuvuuksia ei ole kuitenkaan julkaistussa tutkimuksessa tarkasteltu kuin korkeintaan pareittain.

\citet{ramsahai_connecting_2020} ei katsauksessaan varsinaisesti mainitse Bayes-verkkoja. Hän kuitenkin graafisten mallien sovelluskohteita tapaturmavakuutuksista kerätyssä datassa. Yhtenä sovelluskohteena hän mainitsee riskejä kuvaavien muuttujien välisten yhteyden kuvaamisen ja ymmärtämisen. Tässä tutkielmassa esitän, että nimenomaan Bayes-verkon ja sen rakenteen oppimiseen tarkoittettujen menetelmien avulla tällaisia riippuvuuksia voidaan kuvata ja laskea työeläkedatasta. Tämä mahdollistaa ymmärryksen saavuttamisen työkyvyttömyysriskiin vaikuttavien muuttujien välisistä riippuvuuksista.

Luvussa \ref{background} esittelen niitä matemaattisia käsitteitä, joita tarvitsemme luvussa \ref{bayes} esittelemääni Bayes-verkon käsitteen ja sen rakenteen oppimiseen käytettävien algoritmien perusteiden ymmärtämiseen. Luvussa \ref{applications} tarkastelen Bayes-verkkojen sovellusmahdollisuuksia työkyvyttömyysriskin mallintamisessa ja ennustamisessa. Koska tutkielmani yhtenä tavoitteena on käytännön sovelluskohteiden löytäminen, esittelen luvussa \ref{software} muutaman hyvin dokumentoidun, avoimesti saatavilla olevan ja aktiivisen kehityksen kohteena olevan ohjelmistokirjaston Bayes-verkkojen käsittelemiseen. 



 

