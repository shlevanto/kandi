\chapter{Johdanto\label{intro}}

Tässä tutkielmassa tarkastelen graafisiin malleihin kuuluvien Bayes-verkkojen sovellusmahdollisuuksia työkyvyttömyysriskin mallintamiseen ja ennustamiseen eläkevakuutusdatasta. 

Väestön vanhetessa työvoiman saatavuuden varmentaminen ja työurien pidentäminen on noussut tärkeäksi tavoitteeksi työelämän kehittämisessä. Työkyvyttömyysriskin hallinta on laaja-alaista toimintaa, jolla pyritään vähentämään työntekijöiden riskiä päätyä työkyvyttömäksi ja joutua työkyvyttömyysetuudelle.
Ennenaikainen työkyvyttömyys ja työkyvyttömyyseläkkeet muodostavat ison kuluerän niin työeläkejärjestelmälle eläkemaksuina, kuin koko yhteiskunnalle menetettynä työpanoksena. Vuonna 2021 työkyvyttömyyseläkkeiden suorat kustannukset eläkemaksuina olivat $2.4$ miljardia euroa \citep{etk_show_me_the_money}. 

Työkyvyttömyyden riskitekijöiden tunnistamisella ja ymmärtämisellä voidaan vaikuttaa riskien toteutumiseen ja ennaltaehkäisemiseen. Työkyvyttömyysriskin mallintamisessa ja ennustamisessa on \citet{gross_machine_2020} mukaan perinteisesti käytetty tilastollisia mentelmiä, mutta viime vuosien aikana työkyvyttömyysriskeihin on alettu soveltaa myös erilaisia koneoppimismalleja. Sekä tilasto-, että koneoppimismalleissa muuttujien välisiä riippuvuuksia ei ole julkaistussa tutkimuksessa tarkasteltu kuin korkeintaan pareittain.

 \citet{ramsahai_connecting_2020} kuvailee muutamia tapoja soveltaa graafisia malleja tapaturmavakuutuksista kerätyssä datassa. Yhtenä graafisten mallien käyttötarkoituksena hän tuo esille riskejä kuvaavien muuttujien välisten yhteyden kuvaamisen ja ymmärtämisen. Bayes-verkon avulla nämä riippuvuudet voidaan kuvata ja laskea, mikä mahdollistaa ymmärryksen saavuttamisen mallin eri muuttujien välisistä riippuvuuksista.

Luvussa \ref{background} esittelen niitä matemaattisia käsitteitä, joita tarvitsemme luvussa \ref{bayes} esittelemääni Bayes-verkon käsitteen ja sen rakenteen oppimiseen käytettävien algoritmien perusteiden ymmärtämiseen. Luvussa \ref{applications} tarkastelen Bayes-verkkojen sovelluskohteita ja sovellusmahdollisuuksia työkyvyttömyysriskin mallintamisessa ja ennustamisessa. Tutkielmani yhtenä tavoitteena on käytännön sovelluskohteiden löytäminen, joten esittelen luvussa \ref{software} muutaman hyvin dokumentoidun, avoimesti saatavilla olevan ja aktiivisen kehityksen kohteena olevan ohjelmistokirjaston Bayes-verkkojen käsittelemiseen. 



 

